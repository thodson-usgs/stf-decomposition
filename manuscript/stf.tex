% Overview:
%   Main TeX file for the project.
%   Subfiles should reside in the sections/ directory;
%   and should have a chapter number, title, and a .tex extension.
%
% Build:
%   $ pdflatex dscore.tex

\documentclass{article}
\usepackage{authblk}
\usepackage{amsmath}
\usepackage{amsfonts}
\usepackage{bm}
\usepackage{graphicx}
\usepackage{lineno}
\usepackage{etoolbox}

% Patch 'normal' math environments:
\newcommand*\linenomathpatch[1]{%
  \cspreto{#1}{\linenomath}%
  \cspreto{#1*}{\linenomath}%
  \csappto{end#1}{\endlinenomath}%
  \csappto{end#1*}{\endlinenomath}%
}

\linenomathpatch{equation}
\linenomathpatch{gather}
\linenomathpatch{multline}
\linenomathpatch{align}
\linenomathpatch{alignat}
\linenomathpatch{flalign}


\newcommand{\E}{\mathbb{E}}
\newcommand{\ts}{\textsuperscript}


\title{STF: A seasonal-trend decomposition using the Fast Fourier transform}

\author[1]{TBD}
\affil[1]{U.S. Geological Survey Central Midwest Water Science Center}
\date{}

\linenumbers

\begin{document}

\maketitle

\section{Abstract}
Hip kids write manuscripts using \LaTex and track changes with Git.
We don't use these tools to be hip, we use them because they're the best,
thereby making us hip.
To do this well, it's best to start each sentence on a new line.
Like this.
And this.
A longer sentence should be broken at a natural point, like a conjunction,
because it will tracking changes in Git much easier.

New paragraphs begin by inserting a blank line.

\section{Introduction}
Scientific writing is very formulaic.
The best way to begin a writings is with an outline, like
\begin{itemize}
  \item idea 1, and
  \item idea 2.
\end{itemize}

\section{Algorithm}
Can we write the algorithm two ways:
once with linear algebra,
and again with Python?

\begin{equation}
  E = mc^2
\end{equation}

\section{Discussion}
Compare STR with other seasonal-trend decompositions.

\section{Conclusion}

\end{document}
